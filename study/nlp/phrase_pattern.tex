\documentclass[12pt, dvipdfmx, default, cjk]{beamer}
\usetheme{Antibes}
\usecolortheme{beaver}
\usefonttheme{structurebold}
\setbeamertemplate{navigation symbols}{}
\setbeamertemplate{footline}[frame number]
%\useinnertheme{rectangles}
%\useoutertheme{smoothbars}

% \mathversion{bold}

\usepackage{graphicx}
\usepackage{amsmath, amssymb, cite, url}
\usepackage{algorithm, algorithmic}

%%%

\title{Phrase Patterns for Text Classification}
\author{speaker: Hirakata Kennai}
\institute[CS]{Univ. of Tokyo, CS}
\date{\today}

\begin{document}

\begin{frame} \titlepage \end{frame}
\begin{frame} \tableofcontents \end{frame}

\section{Introduction}

\begin{frame}{読んだ論文}
  ``Learning Phrase Patterns for Text Classification''

  Author: Bin Zhang+

  \url{http://ieeexplore.ieee.org/xpl/articleDetails.jsp?arnumber=6457440}
\end{frame}

\begin{frame}{導入}

テキスト分類
のための素性として
n-gram
が通常使われる

ある程度の精度は達成されてる

ドメインに特化してしまい一般性がない

n-gramとそのまま置き換える素性として、phrase pattern がある
\end{frame}

\begin{frame}{先行研究}
  \begin{itemize}
    \item Wiebe+, 2005

      文章のsubjectを教師ナシで推定する.
      これは目的語を含んだフレーズパターンで分類した。

    \item Sun+, 2007

      第二外国語学習者の書いた誤文法を検出。

    \item Thur and Davidov, 2010

      TwitterやAmazonレビューから「皮肉」な文を検出

    \item Zhang+, 2010

      Speaker role
  \end{itemize}
\end{frame}

\section{Phrase pattern}

\subsection{Phrase pattern}

\begin{frame}{素 phrase pattern}
  文を語の列 $[w_1 \cdots w_n]$ とみなす.
  これに対して phrase pattern とは、
  語の列
  $[u_1 \cdots u_m]$
  と定める.

  phrase pattern が 文にマッチするとは、
  subsequence の関係にあること

  \begin{align*}
    \forall i . ~ u_i = & w_{j_i} \\
    i_1 < i_2 \implies & j_{i_1} < j_{i_2}
  \end{align*}

\end{frame}

\subsection{拡張版 phrase pattern}

\begin{frame}[fragile]{phrase pattern with word classes}
  語の列でなく、word class も利用したい

  word class としては、POSとかpolarity とか
  (個数を制限しない)

  word \texttt{w} の
  class として(文脈に依存して) \texttt{\{c1 .. cn\}} があるとき、

  \begin{verbatim}
  w -> {w, c1 .. cn}
  \end{verbatim}

  という拡張を、文とphrase pattern に対して適用する.
\end{frame}

\begin{frame}
  文 $[w_1 \cdots w_n]$ ($w_i$はトークンとクラスの集合)

  phrase pattern $[u_1 \cdots u_n]$ (同様に集合の列)

  マッチすることの定義は以下のように
  \begin{align*}
    \forall i . ~ u_i \subseteq & w_{j_i} \\
    i_1 < i_2 \implies & j_{i_1} < j_{i_2}
  \end{align*}
\end{frame}

\subsection{Learning patterns}
\begin{frame}{パターンの学習}
  コーパス $D$ から、意味のありそうなパターンの学習アルゴリズムとして
  \begin{itemize}
    \item PrefixSpan
    \item CloSpan
  \end{itemize}
  などがある.
  ここでは一つ目を紹介して、これを改良する.
\end{frame}

\subsection{PrefixSpan}

\begin{frame}{PrefixSpan (Pei, Han+, 2001)}
  コーパス \texttt{D} (文の集合) から、頻度が閾値 \texttt{f} を上回るようなパターンを、
  頭から一つずつ word or class を追加していくことで得る

  パターン $[a]$ が得られたら、
  それを伸ばすような$[a,X]$ も試すことで、
  パターンを学習していく.

  \center
  \includegraphics[width=0.5\textwidth,bb=0 0 328 184]{a.png}
\end{frame}

\begin{frame}[fragile]

  コーパス $D$,
  閾値 $f$
  に対して
  PrefixSpan($D, \rho=[~]$)

  \begin{algorithm}[H]
    \begin{algorithmic}[1]
      \REQUIRE $D$ is a corpus and $\rho$ is a prefix pattern
      \STATE $P \leftarrow \emptyset$
      \FOR{word or class $a$ in $D$}
        \STATE $\rho' \leftarrow $ append$(\rho, a)$
        \IF{matchFreq$(D, \rho') >= f$}
        \STATE $P = P \cup \{\rho'\}$
        \STATE $D' \leftarrow $ $\rho'$-project($D$)
        \STATE $P' = $ PrefixSpan$(D', \rho')$
        \STATE $P = P \cup P'$
        \ENDIF
      \ENDFOR
    \end{algorithmic}
    \caption{PrefixSpan($D, \rho$)}
    \label{alg:seq}
  \end{algorithm}

  $\rho$-project は、パターン$\rho$にマッチする文だけ抽出する射影

\end{frame}

\begin{frame}
  長さ$n$のパターンを見て、長さ$n+1$のパターンを見て$\dots$とやると計算効率が悪いので実際は、後ろに追加した$1$つだけ見ればよい。

\end{frame}

\begin{frame}
  \begin{algorithm}[H]
    \begin{algorithmic}
      \STATE $D' \leftarrow \{\}$
      \STATE $[w_1 \dots w_m] = \rho$
      \FOR{sentence $[b_1 \dots b_n] \in D$}
      \IF{pattern $\rho$ matches this sentence}
      \STATE find indecies $j$ such that
      \STATE $w_1 = b_{j_1} \dots w_m = b_{j_m}$
      \STATE $D' \leftarrow D' \cup \{[b_{j_m + 1} \dots b_m]\}$
      \ENDIF
      \ENDFOR
    \end{algorithmic}
    \caption{$\rho$-project($D$)}
  \end{algorithm}
  射影で、マッチより後ろ部分だけを抽出することで、
  PrefixSpan で、パターンの頻度を確認するときに、
  新たに追加した $a$ だけを、見ればいい.

\end{frame}

\def\given#1#2{p(#1\vert#2)}

\begin{frame}{改良版PrefixSpan}
  尺度として頻度を用いたが、
  分類器において、相互情報量が良い尺度になりうる.

  あるパターンについて,
  マッチするかどうか $X = 0, 1$,
  文書のクラス $Y = 0 \dots K$

  \begin{align*}
    I(X; Y) & = \sum_{x,y} p(x,y) \log \frac{p(x,y)}{p(x) p(y)} \\
    & = \sum_{x,y}
    \given x y
    \log
    \frac{\given x y}
    {
      \sum_{y'} \given{x}{y'} p(y')
    }
  \end{align*}

\end{frame}

\begin{frame}{相互情報量の上限}
  パターン$\rho$がマッチするかどうか$X$,
  $\rho$を伸ばして得たパターン$\rho'$がマッチするかどうかを$XE$と書くと

  \[ \given{XE=1}{y} \le \given{X=1}{y} \]

  とあるから、次のような上限が存在する
  (前頁のIは$\given x y $について上に凸!?)

  \[ \max I(XE; Y) \]

  つまり、パターン$\rho$について、
  パターンを伸ばすことで増価しうる相互情報量の上限は予め算出できる
\end{frame}

\subsection{改良版Prefix Span}
\begin{frame}{改良版Prefix Span}
  尺度を 頻度$\ge f$ から、
  相互情報量を使うように書き換える

  \begin{algorithm}[H]
    \begin{algorithmic}
      \STATE $P \leftarrow \emptyset$
      \FOR{$a$ in $D$}
      \STATE $\rho' \leftarrow $append($\rho, a$)
        \IF{$\rho'$の相互情報量が閾値以上なら}
          \STATE $P \leftarrow P \cup \{\rho'\}$
          \ELSIF{$\rho'$の相互情報量の上限が閾値以上なら}
          \STATE $D' \leftarrow \rho'$-project($D$)
          \STATE $P' \leftarrow $ExtendedPrefixSpan($D', \rho'$)
          \STATE $P \leftarrow P \cup P'$
        \ENDIF
      \ENDFOR
    \end{algorithmic}
    \caption{ExtendedPrefixSpan($D, \rho$)}
  \end{algorithm}

\end{frame}

\section{Word Classes}

\begin{frame}
  実際に使う word class は以下の通り
  \begin{itemize}
    \item Lemma
    \item Word shape
    \item POS
    \item NE
    \item LIWC
    \item subjectivity lexicon
    \item manual
    \item automatic
  \end{itemize}
\end{frame}

\begin{frame}{Lemma}
  語の標準系を取り出す

  \texttt{\{go, goes, going, went gone\} $\rightarrow$ go}

  tool: NLTK WordNet lemmatizer
\end{frame}

\begin{frame}{POS}
  tool: \texttt{Stanford log-linear POS tagger}

  English models and trained models for Arabic, Chinese, French, Spanish, and German
\end{frame}

\begin{frame}[fragile]{Named entity (NE)}

  テキスト分類に於いてはこれが重要ということになっている

\begin{verbatim}
(sentence, word) ->
class ({Location, Person, Organization, Time, Date})
\end{verbatim}

Stanford conditional random field-based NE recognizer (NER)
なるものが良いって。
\end{frame}

\begin{frame}{LIWC dictionary}

  Linguistic Inquiry and Word Count (LIWC)
  は、単語を64の感情に関するクラスに分類する
  
  Facebookが使ってた

  文脈に依存せず、一つの単語について分析する。

  \url{http://www.liwc.net/tryonline.php}

  完全版は \$89.95 で使える
\end{frame}

\begin{frame}[fragile]{MPQA subjectivity lexicon}

  MPQAさんがGNU GPL の元で配布してる辞書

  単語とその品詞から引く形になっている

\begin{verbatim}
(word, POS) -> class
\end{verbatim}

8222項目が登録されてる

\begin{verbatim}
e.g.
type=weaksubj len=1 word1=dominate pos1=verb
stemmed1=y priorpolarity=negative
\end{verbatim}

\end{frame}

\begin{frame}[fragile]{manual}
  そのトピックについて詳しい人間が手作業で、
  そのクラスに属する単語をリストアップしていく.

  あとの実験で使われたものでは

  \begin{verbatim}
AGREEMENT = [right, agree, true]
DISAGREEMENT = [doubt, inappropriate]
ALIGNMENT = AGREEMENT ++ DISAGREEMENT
MODAL = [could, should]
NEGATIVE_DISCOURSE_ORDER =
  [however, but, nevertheless]
  \end{verbatim}

\end{frame}

\begin{frame}{automatic}

Brown+, 1992
"Class-based n-gram models of natural language"
の手法を用いる

1次マルコフモデルを使った、
word のクラスタリング

クラスタ数 = 10, 100, 1000

\end{frame}

\section{実験}

\begin{frame}{実験}

  n-gram (と他の素性) ではそれなりに難しいタスク

  \begin{itemize}
    \item Speaker role
    \item Alignment move
    \item Authority claim
  \end{itemize}

  Baseline を n-gram (3-gramまでに制限) と他
  とするとき、
  pattern (長さ3に制限) と他
  でやってみる

  分類は 最大エントロピー法

  5-fold cross validation で精度またはF値を出す
\end{frame}

\subsection{Speaker role}
\begin{frame}{Speaker role}
  ニュースショー(音源)から、一つのセリフを発した人間の役割 (Host, Guest, Voice bite) を推定する

  Liu+ 80\%
\end{frame}

\begin{frame}{data}
  \begin{itemize}
    \item 48 English talks
    \item 90 Mandarin talks
  \end{itemize}
  の録音に対して、

  \alert{REF} (Reference human transcripts) と
  \alert{ASR} (automatic speech recognition) output
(using SRI Decipher ASR system)
を対象にする.

ASR は、結構間違える.
英語については22.8\%
北京語については38.6\%
くらい、単語/文字を誤る.

$\kappa = 0.67 / 0.78$
\end{frame}

\begin{frame}{Result - English}
\begin{columns}
 \begin{column}{0.63\textwidth}
   \begin{table}[h]
     \begin{tabular}{l|l|l} \hline
       pattern + word class  & Ref.     & ASR \\ \hline
       n-gram (no pattern)   & 85.8\%   & 85.0\% \\
       pattern w/out class & 86.9     & 85.6 \\
       w/ lemma     & 86.8               & 85.4 \\
       w/ POS       & 86.2               & 85.8 \\
       w/ NE        & 86.9               & 84.7 \\
       w/ LIWC      & 86.0               & \alert{85.9} \\
       w/ MPQA      & 86.5               & \alert{85.9} \\
       w/ automatic & \alert{87.1}${^*}$ & 85.6 \\ \hline
     \end{tabular}
   \end{table}
 \end{column}
 \begin{column}{0.3\textwidth}
   Ref に対して、ASR もそこまで悪くない

   n-gram もそんな悪くないんだよね
 \end{column}
 \end{columns}
\end{frame}

\begin{frame}{Result - Chinese}
  \begin{table}[h]
    \begin{tabular}{l|l|l} \hline
      pattern$+$word class & Ref. & ASR \\ \hline
      n-gram               & 84.6 & 70.2 \\
      pattern w/ no class  & \alert{85.8} & \alert{77.8} \\
      w/ POS                  & 84.8 & 74.5 \\
      w/ automatic            & 85.7 & 77.2 \\ \hline
    \end{tabular}
  \end{table}
  中国語はいくつかの素性が使えないからしょうがない。

  あとPOSが使い物になってないのが意外。
\end{frame}

\subsection{Alignment move}
\begin{frame}{Alignment move}
  ネット上の議論において
  ある発言が趣旨に
  同意してるのか(positive)
  反対してるのか(negative)
  を見る

  neutralはない
\end{frame}

\begin{frame}{data}
実験で使うのは
Wikipedia talk page

\begin{itemize}
  \item 211 English pages and 
  \item 225 Chinese pages
\end{itemize}

$\kappa = 0.50 / 0.53$

いくつかの文は pos, neg 両方を含む

あるアノテータがposをつけて、あるアノテータがnegをつけたようなものを、
両方あるものとして、pos+neg というラベルにする

分類は、
\alert{pos/none},
\alert{neg/none}
の2つの分類器を作ってunionをとる
\end{frame}

\begin{frame}{Result - English (F-score)}
  \begin{table}[h]
    \begin{tabular}{l|l|l} \hline
       pattern + word class & Ref.   & ASR \\ \hline
       n-gram (no pattern)  & 38.1\% & 38.8\% \\
       pattern w/out class  & 40.5   & 38.9 \\
       w/ lemma      & 40.2 & 38.8 \\
       w/ word shape & 40.0 & 39.3 \\
       w/ POS        & 39.0 & 38.6 \\
       w/ NE         & 40.5 & 38.9 \\
       w/ LIWC       & 39.0 & 38.7 \\
       w/ MPQA       & 39.2 & \alert{40.5} \\
       w/ manual     & \alert{40.8} & 39.4 \\
       w/ automatic  & 40.7 & \alert{40.5} \\ \hline
    \end{tabular}
  \end{table}
\end{frame}

\begin{frame}{Result - Chinese (F-score)}
  \begin{table}[h]
    \begin{tabular}{l|l|l} \hline
       pattern + word class & Ref.   & ASR \\ \hline
       n-gram (no pattern)  & 26.7\% & 29.7\% \\
       pattern w/out class  & 31.2   & 31.2 \\
       w/ POS       & 32.7         & 30.7 \\
       w/ manual    & \alert{33.9} & \alert{31.5} \\
       w/ automatic & 30.9         & 30.6 \\ \hline
     \end{tabular}
   \end{table}

   基本的に manual が強い
\end{frame}

\subsection{Authority claim}
\begin{frame}{Authority claim}
  \begin{quote}
    showing her knowledge or experience with respect to a topic, or using
    some external evidence to support herself
  \end{quote}

  \begin{itemize}
    \item \alert{forum} claim: フォーラム内のソース(発言)を引用する
    \item \alert{external} claim: 外のソースを引用する
  \end{itemize}

  引用を見るだけだからUnigramで実際けっこう良い (Marin+, 2010)
\end{frame}

\begin{frame}{data}
  \begin{itemize}
    \item 339 English pages and
    \item 225 Chinese pages
  \end{itemize}

  発言ごとに、\alert{forum} / \alert{external} authority claim であるかどうか.

  $\kappa = 0.59 / 0.73$

  データは疎である.
  authority claim は全体の 20 \% だった
\end{frame}

\begin{frame}{Result - English (F-score)}
  \begin{table}[h]
    \begin{tabular}{l|l|l} \hline
       pattern + word class & Ref.   & ASR \\ \hline
       n-gram (no pattern)  & \alert{49.5}\% & 46.5\% \\
       pattern w/out class  & 47.7   & 46.0 \\
       w/ lemma      & 48.0 & 46.7 \\
       w/ word shape & 48.2 & 45.8 \\
       w/ POS        & 48.6 & 45.1 \\
       w/ NE         & 47.7 & 46.0 \\
       w/ LIWC       & 48.9 & 46.5 \\
       w/ MPQA       & 47.8 & 45.5 \\
       w/ manual     & 48.0 & \alert{46.8} \\
       w/ automatic  & 48.9 & 46.1 \\ \hline
     \end{tabular}
   \end{table}
 \end{frame}

\begin{frame}{Result - Chinese (F-score)}
  \begin{table}[h]
    \begin{tabular}{l|l|l} \hline
       pattern + word class & Ref.   & ASR \\ \hline
       n-gram (no pattern)  & 32.2 & 32.3 \\
       pattern w/out class  & 31.8 & 33.5 \\
       w/ POS        & \alert{34.3} & \alert{40.3} \\
       w/ manual     & 30.3 & 37.9 \\
       w/ automatic  & 31.4 & 35.6 \\ \hline
     \end{tabular}
   \end{table}
\end{frame}

\section{まとめ}
\begin{frame}{まとめ}
  \begin{itemize}
    \item
      基本的には

      n-gram
      $\rightarrow$
      素phrase pattern
      $\rightarrow$
      phrase pattern with word classes

      で強くなる

    \item word class は利用可能なら manual が強い
    \item Speech role のASRで見たように、訓練データに頑強性がある
      \pause
    \item この実験は長さ3に制限していたが本気を出したバージョンを見たかった

  \end{itemize}
\end{frame}
\end{document}
